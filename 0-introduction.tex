\begin{refsection}

Depuis sa prédiction en 1934 \parencite{breit_1934}, le processus de création de paires électron-positron par collision de deux photons réels (processus Breit-Wheeler linéaire, BWL) n'a jamais été observé directement en laboratoire, et ce en dépit de son importance fondamentale en électrodynamique quantique et en astrophysique des hautes énergies.

Son rôle en électrodynamique quantique est en effet de première importance, puisqu'il constitue l'un des mécanismes les plus simples décrits par cette théorie (processus d'ordre 2), et sa section efficace est largement utilisée pour calculer la probabilité de création de paires dans la collision de particules chargées, interprétée comme la collision de photons virtuels \parencite{budnev_1975}.

Il est aussi supposé jouer un rôle important dans l'opacité de l'Univers aux photons de hautes énergies, car ces derniers pourraient être absorbés lors de leur propagation en collisionnant avec des photons émis par divers objets, comme par exemple les photons optiques émis par les étoiles, le rayonnement radio émis par les galaxies, voire même avec le fond diffus cosmologique pour les photons de plusieurs centaines de TeV \parencite{diehl_2001, nikishov_1961, gould_1967a, ruffini_2010}. Ce processus pourrait également être à l'origine de jets de paires au voisinage de trous noirs supermassifs \parencite{bonometto_1971} ou de pulsars \parencite{zhang_1998}.

Malgré cela, son observation directe en laboratoire reste toujours à être démontrée depuis les années 1930, et ce principalement à cause de l'absence de sources de photons $\gamma$ de suffisamment haut flux dans la gamme d'énergie du MeV. Néanmoins, le développement des lasers de haute puissance et haute intensité peut désormais permettre d'envisager d'effectuer cette expérience en laboratoire.

En effet, en 2014, les auteur$\cdot$es \cite{pike_2014} ont proposé un schéma expérimental visant à détecter des positrons produits par ce processus dans la collision de photons de plusieurs GeV (produits dans la matière via le processus Bremmstrahlung par un faisceau d'électrons accéléré par laser) avec des photons de quelques centaines d'eV (produits dans un hohlraum éclairé par un laser de haute énergie). Plusieurs autres propositions expérimentales ont alors été publiées depuis \parencite{ribeyre_2016, drebot_2017, yu_2019, wang_2020, he_2020, golub_2020}, dont un schéma de collision de deux sources de photons identiques dans la gamme du MeV \parencite{ribeyre_2016} qui constituera le cadre privilégié de cette thèse.

Dans l'article présentant cette proposition, une étude bibliographique et un modèle théorique permirent d'établir que les sources produites dans l'interaction laser-matière par les processus Bremsstrahlung et Compton inverse multi-photon seraient a priori crédibles pour l'observation du processus BWL en laboratoire. Ce modèle théorique permet d'estimer le nombre de paires produites dans la collision de deux faisceaux de photons, en prenant en compte la divergence des sources, leur distance de collision, et l'énergie totale contenue dans chaque faisceau de photons, mais ne considère néanmoins pas la taille initiale des sources, ni leur durée, l'effet de l'angle de collision, ou encore de la distribution en énergie des photons. Deux autres articles \parencite{ribeyre_2017, ribeyre_2018} publiés les années suivantes permirent d'étudier la cinématique des positrons créés par le processus BWL, et démontrent que sous certaines conditions il est possible de les diriger préférentiellement dans une direction donnée. Un code de collision de photons a aussi été développé \parencite{jansen_2018}, et celui-ci permet d'estimer à la fois le nombre et la cinématique des paires produites via le processus BWL dans la collision de faisceaux de photons.

Ainsi, l'objectif de cette thèse théorique et numérique, débutée en octobre 2017 et qui s'inscrit dans le projet ANR TULIMA, est de préparer le dimensionnement d'une future expérience pour la détection du processus BWL en laboratoire. 
Les travaux menés dans ce cadre peuvent être résumés en 4 grands axes, qui seront détaillés dans les 6 chapitres composant ce manuscrit :
\begin{enumerate}
    \item Tout d'abord, un important effort de contextualisation a été fourni. 
    Le processus Breit-Wheeler linéaire a été resitué dans la thématique de la collision de photons, et plus généralement de l'électrodynamique quantique (chapitre \ref{chap:1-particules}). Plusieurs processus de production de photons ont ensuite été considérés (chapitre \ref{chap:1-particules}), et une étude bibliographique a permis de dégager les différentes possibilités pour la production de photons de quelques MeV par laser (chapitre \ref{chap:2-laser}). Les différentes propositions d'expériences pour l'étude de la création de paires via le processus BWL sont présentées et discutées au chapitre \ref{chap:3-methodes_exp}, ainsi que les stratégies qui seront spécifiquement suivies dans le cadre de cette thèse.
    
    \item Une chaîne de simulations a aussi été développée pour l'étude numérique de la production de photons $\gamma$ par laser, en se concentrant particulièrement sur les sources Bremsstrahlung. Celle-ci comprend le code Particle-In-Cell Smilei \parencite{derouillat_2018} permettant de simuler l'interaction laser-plasma et l'accélération d'électrons par laser, une application Monte Carlo Geant4 \parencite{agostinelli_2003} développée dans le cadre de cette thèse et nommée gp3m2 \parencite{gp3m2} permettant d'étudier la propagation de particules dans la matière et la production de photons $\gamma$ via le processus Bremsstrahlung, ainsi que le code TrILEns \parencite{jansen_2018} permettant d'étudier la création de paires BWL dans la collision de faisceaux de photons. Un module d'analyse d'espace des phases nommé p2sat \parencite{p2sat} a aussi été développé, et permet d'analyser les résultats de ces simulations et de transférer les données entre ces différents codes. Ces aspects sont discutés au chapitre \ref{chap:4-methodes_simu}.
    
    \item Des réflexions théoriques ont également permis d'améliorer la précision du modèle de \cite{ribeyre_2016} pour les estimations du nombre de paires produites, en prenant en compte les aspects les plus importants de la collision de deux faisceaux produits par laser, que sont leurs propriétés géométriques (durée, angle de divergence, ...), ainsi que leurs distributions en énergie. Cette approche, présentée au chapitre \ref{chap:5-opti_theorique}, se base principalement sur le formalisme de la luminosité \parencite{herr_2006}. Une adaptation du modèle de \cite{ribeyre_2017} a aussi été proposée, pour tenter d'estimer les caractéristiques de positrons produits par la collision de deux faisceaux de photons avec des distributions en énergies larges.
    
    \item Enfin, la chaîne de simulations développée dans le cadre de cette thèse a été utilisée pour optimiser la production de sources de photons autour du MeV par Bremsstrahlung, à partir de lasers d'énergies dans la gamme du Joule et avec un taux de répétition autour du Hertz. Nous nous sommes principalement concentrés sur l'interaction de lasers d'intensités $10^{19}$ à $10^{21} ~ \si{\W\per\cm^2}$ avec des fils nanométriques ou micrométriques de différentes dimensions. À l'aide de simulations Particle-In-Cell 2D, nous avons réussi à dégager deux tendances quant au régime d'interaction laser-plasma, en fonction de l'épaisseur de ces fils. Les sources d'électrons produites ont ensuite été injectées dans différents convertisseurs, dont l'épaisseur a été optimisée pour maximiser la brillance à 1 MeV des sources de photons. Ces dernières ont enfin été transférées dans l'application de collision de photons TrILEns, afin d'estimer le nombre de paires BWL qui seraient susceptibles d'être produites par la collision de faisceaux de ce type, avec un taux de répétition de l'ordre du Hertz. Les résultats de cette étude sont présentés et commentés au chapitre \ref{chap:6-opti_numerique}.
\end{enumerate}

Chaque chapitre peut se lire de manière relativement indépendante, et un résumé est présent à chaque début de chapitre, où des références complémentaires sont indiquées. Un glossaire à la fin de ce manuscrit reprend les principales abréviations et définit quelques termes techniques, ainsi que les constantes physiques utilisées. Les quantités physiques sont données en unités pratiques, et les unités dans les équations sont indiquées entre crochets. Les images créées pour ce manuscrit sont disponibles au format svg sous licence CC-BY à l'adresse \href{https://github.com/lesnat/these}{https://github.com/lesnat/these}.

\newpage
\printbibliography[heading=subbibintoc]
\end{refsection}