\uchapter{Glossaire}
\section*{Abréviations et terminologie}

Les abréviations utilisées dans le cadre de cette thèse sont :
\begin{itemize}
    \item \textit{BWL} : Breit-Wheeler linéaire
    \item \textit{QED} : Électrodynamique quantique (Quantum Electrodynamics)
    \item \textit{CM} : Centre d'inertie, centre des impulsions (Center of Momentum)
    \item \textit{PIC} : Particle-In-Cell
    \item \textit{MC} : Monte Carlo
    \item \textit{FDE} : Fonction de distribution en énergie
    \item \textit{CIL} : Compton inverse linéaire
    \item \textit{LMC} : Laser dans un micro-canal
    \item \textit{Brem} : Bremsstrahlung
\end{itemize}

Dans cette thèse, nous utilisons le terme "température" pour désigner la quantité $k_B ~ T$, où $k_b$ est la constante de Boltzman et $T$ la température physique mesurée en Kelvin.
L'appellation "densité" fait référence à un nombre par unité de volume et est notée $n_\alpha$ pour l'espèce $\alpha$, alors qu'une masse par unité de volume est appelé "masse volumique" et est notée $\rho$.

\section*{Constantes physiques}
Les constantes physiques utilisées sont :
\begin{itemize}
    \item $c$ : la vitesse de la lumière dans le vide
    \item $\alpha$ : la constante de structure fine
    \item $\hbar$ : la constante de Planck réduite
    \item $r_e$ : le rayon classique de l'électron
    \item $m_e$ : la masse de l'électron
    \item $\epsilon_0$ : la permittivité diélectrique du vide, ou constante électrique
    \item $\mu_0$ : la perméabilité magnétique du vide, ou constante magnétique
\end{itemize}

\uchapter{Publications et contributions}
\section*{Publications}

\noindent L. Esnault, E. d'Humières, A. Arefiev, X. Ribeyre, \textit{Energy distribution effects in electron-positron pair production by real photon beam collisions}, soumis à New Journal of Physics.

\section*{Posters}

\noindent L. Esnault, X. Ribeyre, E. d'Humières, S. Jequier, S. Hulin, J.L. Dubois, V. Tikhonchuk, L. Lancia, J.R. Marques. Production de photons gamma collimatés et création de paires électron-positron. Congrès SFP Plasmas 2018.

\noindent L. Esnault, X. Ribeyre, E. d'Humières, Production de photons gamma collimatés pour la création de paires électron-positron, Journée de l'école doctorale SPI, Université de Bordeaux, 2019.

\section*{Présentations orales}
\noindent L. Esnault, X. Ribeyre, E. d’Humières, J. L. Dubois, D. Khaghani, S. Jequier, P. Lageyre, L. Lancia, J.R. Marques, Production of collimated gamma-ray beam for electron-positron pair creation. 46th European Physics Society Conference on Plasma Physics, 2019, \url{ocs.ciemat.es/EPS2019ABS/pdf/O4.206.pdf}.

\noindent Les séminaires envisagés pour l'été 2020 ont été annulés.

\section*{Interventions de vulgarisation}

\noindent Participation à la Nuit européenne des chercheur$\cdot$es (poster), 2018.

\noindent Présentation dans le cadre du festival Pint Of Science, 2019.
