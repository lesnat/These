\begin{refsection}
\section{Résumé et conclusion de ces travaux}

En dépit de son importance fondamentale en électrodynamique quantique et en astrophysique des hautes énergies, le processus de création de paires électron-positron par collision de deux photons réels, appelé processus Breit-Wheeler linéaire (BWL), n'a jusqu'à présent jamais été observé en laboratoire depuis sa prédiction en 1934, et ce principalement à cause de l'absence de sources de photons de suffisamment haut flux dans la gamme du MeV.

Néanmoins, depuis 2014 plusieurs schémas expérimentaux ont pu être proposés pour effectuer cette expérience en laboratoire \parencite{pike_2014, ribeyre_2016, drebot_2017, yu_2019, wang_2020, he_2020, golub_2020}.
La création de faisceaux de photons de l'ordre du MeV y fait toujours intervenir des lasers, que ce soit pour accélérer des électrons, pour constituer une source dense de photons de basse énergie ou une source de champ électromagnétique intense.
Dans le cadre de cette thèse, nous nous sommes principalement concentrés sur l'interaction de sources identiques et directionnelles de photons d'énergies autour du MeV, en suivant la proposition de \cite{ribeyre_2016}. 

Les travaux effectués dans le cadre de cette thèse se sont concentrés sur 4 axes majeurs :

\begin{enumerate}
    \item Tout d'abord, un important effort de contextualisation a été fournis. 
    Nous avons resitué la place du processus BWL en électrodynamique quantique, et explicité ses liens avec d'autres processus déjà détectés en laboratoire. 
    Deux stratégies différentes pour la création de positrons BWL par des lasers de haute puissance ont ensuite été évoquées. En effet, les lasers de milieu amplificateur \textit{Nd:verre} ont typiquement une énergie de la gamme du kilo-Joule et un taux de répétition de l'ordre du tir par heure \parencite{danson_2019}. La stratégie à mettre en place pour une expérience visant à détecter le processus BWL via ce type de lasers nécessiterait donc de produire et détecter un nombre important de positrons en peu de tirs. Au contraire, les lasers de milieu amplificateur \textit{Ti:Sa} ont typiquement une énergie dans la gamme du Joule et un taux de répétition de l'ordre du tir par seconde \parencite{danson_2019}, et permettraient d'effectuer un plus grand nombre de tirs par campagne expérimentale.
    Nous avons montré que ces lasers pourraient permettre de produire des photons de plusieurs MeV via au moins trois processus (Compton inverse linéaire, Compton inverse multi-photon et Bremsstrahlung) et quatre types de schéma expérimental (faisceau d'électron contra-propagatif à un laser d'intensité non relativiste ou relativiste, faisceau d'électrons co-propagatif à un faisceau laser d'intensité relativiste, ou faisceau d'électrons accéléré par laser injecté dans un matériau de numéro atomique élevé). 
    Des estimations menées à partir du modèle de \cite{ribeyre_2016} nous ont aussi indiquées que l'interaction de faisceaux de photons $\gamma$ produits via le processus Bremsstrahlung par des lasers d'énergie dans la gamme du Joule pourrait permettre de créer de l'ordre de quelques paires BWL par tir, et serait adaptée à une expérience menée à taux de répétition important.

    \item Afin d'étudier cette possibilité plus précisément, nous avons mis en place une chaîne de simulations permettant de traiter l'aspect très multi-physique du problème. Le code \textit{Particle-In-Cell} Smilei \parencite{derouillat_2018} est utilisé pour simuler l'interaction de l'impulsion principale avec la cible ainsi que l'accélération d'électrons par laser. Ces électrons sont ensuite injectés dans un code \textit{Monte Carlo} pour simuler leur propagation dans la matière et la production des photons $\gamma$ via le processus Bremsstrahlung. La création de paires électron-positron par le processus BWL peut enfin être estimée en injectant ces photons dans le code TrILEns \parencite{jansen_2018}. Dans ce cadre, une application \textit{Monte Carlo} Geant 4 \parencite{agostinelli_2003} nommée gp3m2 a été développée et est disponible en open-source \parencite{gp3m2}. Une réflexion a aussi été menée pour minimiser la perte d'informations à l'interface entre ces codes, et plusieurs méthodes permettant de manipuler et visualiser l'espace des phases de particules ont été implémentés dans un module Python nommé p2sat, lui aussi développé dans le cadre de cette thèse et disponible en open-source \parencite{p2sat}.

    \item Des développements théoriques ont aussi été entrepris, et ceux-ci permettent d'estimer à la fois le nombre total et les caractéristiques cinématiques des positrons créés par BWL dans l'interaction de faisceaux de photons avec des distributions en énergies larges (Bremsstrahlung, Compton inverse linéaire et Compton inverse multi-photon). Le nombre de positrons est calculé comme le produit de deux termes que sont : la luminosité, prenant en compte les aspects géométriques de la collision des faisceaux ; et une section efficace intégrée, prenant en compte ses aspects énergétiques. Le concept de la luminosité est très utilisé dans la communauté des collisionneurs de particules \parencite{herr_2006}, et la notion de section efficace intégrée est similaire aux calculs menés dans la double approximation de photon équivalent \parencite{kessler_1974}. Un modèle basé sur les résultats de \cite{ribeyre_2017} a aussi été développé pour estimer la cinématique des positrons produits.

    \item Enfin, la chaîne de simulations a été utilisée pour étudier la collision de sources de photons $\gamma$, produites par les processus Bremsstrahlung et Compton inverse multi-photon dans l'interaction de lasers d'intensités $I_0=10^{19}$, $10^{20}$ et $10^{21} ~ \si{\W\per\cm^2}$ avec des cibles bi-couches composées d'un substrat et d'un absorbant structuré avec des fils micrométriques ou nanométriques. Pour les sources Bremsstrahlung, les électrons accélérés par ces lasers ont été injectés dans des convertisseurs en platine, dont l'épaisseur a été optimisée pour maximiser la brillance crête à 1 MeV des faisceaux de photons produits. Ces photons ont alors été transférés dans le code TrILEns pour estimer le nombre de paires BWL qu'il serait envisageable de produire dans la collision de deux de ces sources. Nous avons montré que, pour une campagne expérimentale de 10 jours et à raison de 10 000 tirs par jour ($\sim 3$ heures à un taux de répétition de 1 Hz), on peut estimer qu'il serait possible de produire de l'ordre de $1$ paire pour deux lasers d'intensité $I_0=10^{19} ~ \si{\W\per\cm^2}$, de l'ordre de $10^2$ paires pour deux lasers d'intensité $I_0=10^{20} ~ \si{\W\per\cm^2}$, ou de l'ordre de $10^5$ paires pour deux lasers d'intensité $I_0=10^{21} ~ \si{\W\per\cm^2}$. Pour le cas de plus haute intensité considérée, ce nombre est suffisamment significatif pour permettre d'envisager cette expérience sur des lasers actuels ou en construction tels que Astra Gemini \parencite{laser_gemini} ou HAPLS \parencite{laser_hapls} par exemple. 
    Des études complémentaires seraient néanmoins nécessaires, pour prendre en compte les effets de la pré-impulsion, de la géométrie PIC 3D, et pour étudier plus précisément le bruit de mesure qui pourrait être très important dans ce type d'expérience. Il serait aussi possible d'étudier des sources produites par Bremsstrahlung dans d'autres types de régimes d'interaction laser-matière, notamment via l'accélération d'électrons dans des jets de gaz.
    Les sources créées via le processus Compton inverse multi-photon produisent quant à elles un nombre de paires BWL plus faibles que les sources Bremsstrahlung pour une intensité $I_0=10^{21} ~ \si{\W\per\cm^2}$, mais exhibent un bruit de mesure a priori beaucoup plus faible. Elles pourraient être à ce titre potentiellement intéressantes, surtout à plus haute intensité où des simulations numériques montrent que leur efficacité peut augmenter significativement. De plus amples investigations seraient néanmoins ici aussi nécessaires.
\end{enumerate}

Ainsi, les travaux entrepris dans le cadre de cette thèse ont permis de dégager plusieurs stratégies pour la production de photons $\gamma$ par laser, et pour la création de paires par collision de deux photons réels. Nous nous sommes principalement concentrés sur la collision de faisceaux de photons identiques de quelques MeV, générés via le processus Bremssstrahlung, et qui pourraient être produits par des lasers intenses d'énergies dans la gamme du Joule avec un taux de répétition de l'ordre du Hertz. Des développements théoriques ont par ailleurs montré que ce type de sources serait a priori assez efficace pour cette application, et serait peu sensible à des variations de conditions expérimentales. Une chaîne de simulations a aussi été développée, et a été appliquée à l'étude de sources de photons $\gamma$ produites dans l'interaction de ce type de lasers avec des cibles micro et nano-structurées accolées à un convertisseur en platine. La collision de ces faisceaux de photons a permis de montrer que le nombre de paires BWL produites pourrait être suffisamment significatif pour pouvoir envisager cette expérience avec des lasers intenses d'énergies de l'ordre de la dizaine de Joules. Des optimisations ultérieures ainsi que des études plus approfondies du bruit de mesure seraient néanmoins nécessaires.


\section{Perspectives}

Au delà de l'application de la création de paires électron-positron par collision de photons, cette thèse s'inscrit plus généralement dans des thématiques de recherche fondamentale très actuelles, comme l'étude la de collision de photons réels \parencite{marklund_2006, chou_2018, takahashi_2019}, ainsi que l'étude de processus d'électrodynamique quantique par le biais de lasers \parencite{dipiazza_2012, zhang_2020}. 
Elle est fortement imprégnée des recherches menées sur la production de particules par laser \parencite{ledingham_2010}, et plus particulièrement sur la production de photons énergétiques \parencite{albert_2016, corde_2013a}.
Les outils numériques utilisés dans ce cadre font aussi appel à des technologies innovantes \parencite{derouillat_2018, agostinelli_2003, jansen_2018, gp3m2, p2sat, fryxell_2000}, qui mis à part TrILEns sont toutes en open-source. 
Les études numériques menées avec ces applications se basent sur des cibles structurées, dont le développement est une thématique de recherche importante pour le domaine de l'interaction laser-plasma \parencite{passoni_2020, prencipe_2017} ainsi que pour d'autres domaines très variés \parencite{kuchibhatla_2007}. 
La production de ces sources de photons $\gamma$ avec un taux de répétition de l'ordre du tir par seconde nécessiterait aussi des technologies de pointe, autant en terme de laser \parencite{danson_2019} que de cibles \parencite{prencipe_2017}. 
La démonstration expérimentale du principe de détection de positrons développé au CELIA conjointement à cette thèse serait aussi une première dans un environnement si bruité.

Les travaux menés dans le cadre de cette thèse pourraient être approfondis ou réadaptés de multiples manières.
Par exemple, les études bibliographiques synthétisées au début de ce manuscrit pourraient servir de base pour envisager d'autres stratégies de production de photons $\gamma$ pour l'étude du processus BWL, et pourraient éventuellement être réadaptées à l'étude d'autres processus d'électrodynamique quantique par laser.
La chaîne de calcul développée dans le cadre de cette thèse est aussi suffisamment modulaire pour pouvoir être appliquée à de nombreuses applications physiques. 
En particulier, l'application Monte Carlo gp3m2 est d'ores et déjà adaptée à la maximisation du nombre de photons produits, ou de l'efficacité de conversion énergétique du laser dans ces photons par exemple, mais pourrait aussi très bien être utilisée pour maximiser la production de positrons dans la matière. Des adaptations ultérieures pourraient facilement permettre d'utiliser cette application pour optimiser la production d'autres types de particules, comme les neutrons, les pions ou les muons par exemple.
Le module d'analyse d'espace des phases p2sat développé durant cette thèse a aussi été construit de façon à pouvoir être utilisé indépendamment de l'origine des données, et est donc nativement adaptable à n'importe quel type de simulation utilisant des espaces des phases de particules, du moment qu'il est possible de lire ses fichiers de sortie en Python.
Le modèle théorique développé dans le cadre de cette thèse pourrait être utile pour mener des estimations d'ordre de grandeurs, afin de pouvoir dimensionner une expérience visant à détecter le processus BWL en laboratoire. La fonction d'ajustement de la section efficace intégrée pourrait être utilisée dans des codes de calcul, afin de s'affranchir de l'étape d'intégration en énergie nécessaire pour prendre en compte des distributions en énergie larges. En considérant d'autres formes de distributions en énergies, ce modèle pourrait aussi être adapté à l'étude de situations astrophysiques, voir à l'étude d'autres processus de collision de particules.
Les sources d'électrons et de photons optimisées dans notre étude numérique pourraient aussi être appliquées dans d'autres contextes. 
Enfin, l'observation directe du processus BWL en laboratoire serait une première, et permettrait d'envisager l'étude expérimentale de prédictions théoriques dans des domaines plus avancés \parencite{hartin_2007, satunin_2018}.

\newpage
\printbibliography[heading=subbibintoc]
\end{refsection}
